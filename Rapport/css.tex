\definecolor{mygreen}{rgb}{0,0.6,0}
\definecolor{mygray}{rgb}{0.5,0.5,0.5}
\definecolor{mymauve}{rgb}{0.58,0,0.82}

\lstset{ %
	backgroundcolor=\color{white},   % choose the background color; you must
	%add \usepackage{color} or \usepackage{xcolor}; should come as last argument
	basicstyle=\footnotesize,        % the size of the fonts that are used for
	%the code
	breakatwhitespace=false,         % sets if automatic breaks should only
	%happen at whitespace
	breaklines=true,                 % sets automatic line breaking
	captionpos=b,                    % sets the caption-position to bottom
	commentstyle=\color{mygreen},    % comment style
	deletekeywords={...},            % if you want to delete keywords from the
	%given language
	escapeinside={\%*}{*)},          % if you want to add LaTeX within your code
	extendedchars=true,              % lets you use non-ASCII characters; for
	%8-bits encodings only, does not work with UTF-8
	frame=single,	                   % adds a frame around the code
	keepspaces=true,                 % keeps spaces in text, useful for keeping
	%indentation of code (possibly needs columns=flexible)
	keywordstyle=\color{blue},       % keyword style
	language=Octave,                 % the language of the code
	morekeywords={*,...},           % if you want to add more keywords to the
	%set
	numbers=left,                    % where to put the line-numbers; possible
	%values are (none, left, right)
	numbersep=5pt,                   % how far the line-numbers are from the
	%code
	numberstyle=\tiny\color{mygray}, % the style that is used for the
	%line-numbers
	rulecolor=\color{black},         % if not set, the frame-color may be
	%changed on line-breaks within not-black text (e.g. comments (green here))
	showspaces=false,                % show spaces everywhere adding particular
	%underscores; it overrides 'showstringspaces'
	showstringspaces=false,          % underline spaces within strings only
	showtabs=false,                  % show tabs within strings adding
	%particular underscores
	stepnumber=1,                    % the step between two line-numbers. If
	%it's 1, each line will be numbered
	stringstyle=\color{mymauve},     % string literal style
	tabsize=2,	                   % sets default tabsize to 2 spaces
	title=\lstname,                   % show the filename of files included with
	%\lstinputlisting; also try
	%n instead of title
	literate= %Gestion UTF-8
  		{á}{{\'a}}1 {é}{{\'e}}1 {í}{{\'i}}1 {ó}{{\'o}}1 {ú}{{\'u}}1
  		{Á}{{\'A}}1 {É}{{\'E}}1 {Í}{{\'I}}1 {Ó}{{\'O}}1 {Ú}{{\'U}}1
  		{à}{{\`a}}1 {è}{{\`e}}1 {ì}{{\`i}}1 {ò}{{\`o}}1 {ù}{{\`u}}1
  		{À}{{\`A}}1 {È}{{\'E}}1 {Ì}{{\`I}}1 {Ò}{{\`O}}1 {Ù}{{\`U}}1
  		{ä}{{\"a}}1 {ë}{{\"e}}1 {ï}{{\"i}}1 {ö}{{\"o}}1 {ü}{{\"u}}1
  		{Ä}{{\"A}}1 {Ë}{{\"E}}1 {Ï}{{\"I}}1 {Ö}{{\"O}}1 {Ü}{{\"U}}1
  		{â}{{\^a}}1 {ê}{{\^e}}1 {î}{{\^i}}1 {ô}{{\^o}}1 {û}{{\^u}}1
  		{Â}{{\^A}}1 {Ê}{{\^E}}1 {Î}{{\^I}}1 {Ô}{{\^O}}1 {Û}{{\^U}}1
  		{œ}{{\oe}}1 {Œ}{{\OE}}1 {æ}{{\ae}}1 {Æ}{{\AE}}1 {ß}{{\ss}}1
  		{ű}{{\H{u}}}1 {Ű}{{\H{U}}}1 {ő}{{\H{o}}}1 {Ő}{{\H{O}}}1
  		{ç}{{\c c}}1 {Ç}{{\c C}}1 {ø}{{\o}}1 {å}{{\r a}}1 {Å}{{\r A}}1
  		{€}{{\euro}}1 {£}{{\pounds}}1 {«}{{\guillemotleft}}1
  		{»}{{\guillemotright}}1 {ñ}{{\~n}}1 {Ñ}{{\~N}}1 {¿}{{?`}}1
}

%Reglages pour les liens
\hypersetup{
  colorlinks=false,
  linktoc=true, %Pour mettre les liens entre la table des matières et les sections
  pdfauthor = {BOYER Benoît},
  pdftitle = {HMIN317 - Moteur de jeux},
  pdfsubject = {Compte-rendu TP2},
}

%Redifinition du pied de page pour le chapitre uniquement
\fancypagestyle{plain}{%
\renewcommand{\footrulewidth}{1pt}
\fancyfoot[L]{BOYER Benoît}
\fancyfoot[C]{}
\fancyfoot[R]{\textbf{Page \thepage \ sur \pageref{LastPage}}}

\renewcommand{\headrulewidth}{0pt}
\fancyhead[R]{}
\fancyhead[C]{}
\fancyhead[L]{}
}

%Reglages pour en-têtes et pieds de page
\pagestyle{fancy}

\renewcommand{\headrulewidth}{1pt}
\fancyhead[R]{HMIN317 - Moteur de jeux}
\fancyhead[C]{}
\fancyhead[L]{\leftmark}

\renewcommand{\footrulewidth}{1pt}
\fancyfoot[L]{BOYER Benoît}
\fancyfoot[C]{}
\fancyfoot[R]{\textbf{Page \thepage \ sur \pageref{LastPage}}}

%Pour afficher et charger des images
\newcommand{\image}[2] {
	\begin{figure}[H]
		\centering
		\includegraphics[width=1.0\textwidth]{#1}
		\caption{#2}
	\end{figure}
}