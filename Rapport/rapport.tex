% Document description
\documentclass[a4paper,11pt]{report}
\usepackage[utf8]{inputenc}
\usepackage[T1]{fontenc}
\usepackage{lmodern}
\usepackage[francais]{babel}
\usepackage{listings}
\usepackage{graphicx} %Pour inclure les images
\usepackage{float} %Pour plus de précision sur le placement
\usepackage{color}
\usepackage[hidelinks]{hyperref} %Pour les liens dans le PDF
\usepackage{fancyhdr} %En-tête + pieds de page
\usepackage{lastpage}

% Metadata
\title{HMIN317 - Moteur de jeux \\ Compte-rendu TP2}
\author{BOYER Benoît}
\date{Septembre 2017}

\definecolor{mygreen}{rgb}{0,0.6,0}
\definecolor{mygray}{rgb}{0.5,0.5,0.5}
\definecolor{mymauve}{rgb}{0.58,0,0.82}

\lstset{ %
	backgroundcolor=\color{white},   % choose the background color; you must
	%add \usepackage{color} or \usepackage{xcolor}; should come as last argument
	basicstyle=\footnotesize,        % the size of the fonts that are used for
	%the code
	breakatwhitespace=false,         % sets if automatic breaks should only
	%happen at whitespace
	breaklines=true,                 % sets automatic line breaking
	captionpos=b,                    % sets the caption-position to bottom
	commentstyle=\color{mygreen},    % comment style
	deletekeywords={...},            % if you want to delete keywords from the
	%given language
	escapeinside={\%*}{*)},          % if you want to add LaTeX within your code
	extendedchars=true,              % lets you use non-ASCII characters; for
	%8-bits encodings only, does not work with UTF-8
	frame=single,	                   % adds a frame around the code
	keepspaces=true,                 % keeps spaces in text, useful for keeping
	%indentation of code (possibly needs columns=flexible)
	keywordstyle=\color{blue},       % keyword style
	language=Octave,                 % the language of the code
	morekeywords={*,...},           % if you want to add more keywords to the
	%set
	numbers=left,                    % where to put the line-numbers; possible
	%values are (none, left, right)
	numbersep=5pt,                   % how far the line-numbers are from the
	%code
	numberstyle=\tiny\color{mygray}, % the style that is used for the
	%line-numbers
	rulecolor=\color{black},         % if not set, the frame-color may be
	%changed on line-breaks within not-black text (e.g. comments (green here))
	showspaces=false,                % show spaces everywhere adding particular
	%underscores; it overrides 'showstringspaces'
	showstringspaces=false,          % underline spaces within strings only
	showtabs=false,                  % show tabs within strings adding
	%particular underscores
	stepnumber=1,                    % the step between two line-numbers. If
	%it's 1, each line will be numbered
	stringstyle=\color{mymauve},     % string literal style
	tabsize=2,	                   % sets default tabsize to 2 spaces
	title=\lstname,                   % show the filename of files included with
	%\lstinputlisting; also try
	%n instead of title
	literate= %Gestion UTF-8
  		{á}{{\'a}}1 {é}{{\'e}}1 {í}{{\'i}}1 {ó}{{\'o}}1 {ú}{{\'u}}1
  		{Á}{{\'A}}1 {É}{{\'E}}1 {Í}{{\'I}}1 {Ó}{{\'O}}1 {Ú}{{\'U}}1
  		{à}{{\`a}}1 {è}{{\`e}}1 {ì}{{\`i}}1 {ò}{{\`o}}1 {ù}{{\`u}}1
  		{À}{{\`A}}1 {È}{{\'E}}1 {Ì}{{\`I}}1 {Ò}{{\`O}}1 {Ù}{{\`U}}1
  		{ä}{{\"a}}1 {ë}{{\"e}}1 {ï}{{\"i}}1 {ö}{{\"o}}1 {ü}{{\"u}}1
  		{Ä}{{\"A}}1 {Ë}{{\"E}}1 {Ï}{{\"I}}1 {Ö}{{\"O}}1 {Ü}{{\"U}}1
  		{â}{{\^a}}1 {ê}{{\^e}}1 {î}{{\^i}}1 {ô}{{\^o}}1 {û}{{\^u}}1
  		{Â}{{\^A}}1 {Ê}{{\^E}}1 {Î}{{\^I}}1 {Ô}{{\^O}}1 {Û}{{\^U}}1
  		{œ}{{\oe}}1 {Œ}{{\OE}}1 {æ}{{\ae}}1 {Æ}{{\AE}}1 {ß}{{\ss}}1
  		{ű}{{\H{u}}}1 {Ű}{{\H{U}}}1 {ő}{{\H{o}}}1 {Ő}{{\H{O}}}1
  		{ç}{{\c c}}1 {Ç}{{\c C}}1 {ø}{{\o}}1 {å}{{\r a}}1 {Å}{{\r A}}1
  		{€}{{\euro}}1 {£}{{\pounds}}1 {«}{{\guillemotleft}}1
  		{»}{{\guillemotright}}1 {ñ}{{\~n}}1 {Ñ}{{\~N}}1 {¿}{{?`}}1
}

%Reglages pour les liens
\hypersetup{
  colorlinks=false,
  linktoc=true, %Pour mettre les liens entre la table des matières et les sections
  pdfauthor = {BOYER Benoît},
  pdftitle = {HMIN317 - Moteur de jeux},
  pdfsubject = {Compte-rendu TP2},
}

%Redifinition du pied de page pour le chapitre uniquement
\fancypagestyle{plain}{%
\renewcommand{\footrulewidth}{1pt}
\fancyfoot[L]{BOYER Benoît}
\fancyfoot[C]{}
\fancyfoot[R]{\textbf{Page \thepage \ sur \pageref{LastPage}}}

\renewcommand{\headrulewidth}{0pt}
\fancyhead[R]{}
\fancyhead[C]{}
\fancyhead[L]{}
}

%Reglages pour en-têtes et pieds de page
\pagestyle{fancy}

\renewcommand{\headrulewidth}{1pt}
\fancyhead[R]{HMIN317 - Moteur de jeux}
\fancyhead[C]{}
\fancyhead[L]{\leftmark}

\renewcommand{\footrulewidth}{1pt}
\fancyfoot[L]{BOYER Benoît}
\fancyfoot[C]{}
\fancyfoot[R]{\textbf{Page \thepage \ sur \pageref{LastPage}}}

%Pour afficher et charger des images
\newcommand{\image}[2] {
	\begin{figure}[H]
		\centering
		\includegraphics[width=1.0\textwidth]{#1}
		\caption{#2}
	\end{figure}
}


% --------------> Document beginning <--------------
\begin{document}

  	  %Pattern de Peter Wilson
  	  \begin{titlepage} % Suppresses displaying the page number on the title page and the subsequent page counts as page 1
	
	  \raggedleft % Right align the title page
	
	  \rule{1pt}{\textheight} % Vertical line
	  \hspace{0.05\textwidth} % Whitespace between the vertical line and title page text
	  \parbox[b]{0.75\textwidth}{ % Paragraph box for holding the title page text, adjust the width to move the title page left or right on the page
		
		  {\Huge\bfseries Compte-rendu TP2 \\[0.5\baselineskip] Game loops et timers}\\[2\baselineskip] % Title
		  {\large\textit{HMIN317 - Moteur de jeux}}\\[4\baselineskip] % Subtitle or further description
		  {\Large\textsc{BOYER Benoît}} % Author name, lower case for consistent small caps
		
		  \vspace{0.5\textheight} % Whitespace between the title block and the publisher
		
		  {\noindent M2 IMAGINA - Septembre 2017}\\[\baselineskip] % Publisher and logo
	  }

  \end{titlepage}
  
    \tableofcontents
	\pagebreak


    \section{Question 1}
    \subsection{Modifier votre TP précédent pour lire une height map}
    Dans un premier temps, on va devoir charger l'image qui va nous servir de height map dans la fonction \texttt{GeometryEngine::initPlaneGeometry(QWidget *qw)}, auquel on aura rajouté le paramètre \texttt{QWidget *qw} pour avoir des fenêtres.
    \lstinputlisting[language=C++, caption=Selection de la height map, firstline=132, lastline=139, firstnumber=132]{../geometryengine.cpp}
    \image{choixHeightmap.png}{La fenêtre de dialogue obtenue}
    Ensuite, on utilise une \texttt{QImage} qui servira à parcourir l'image pour récupérer la valeur des pixels pour les assigner à une hauteur.
    
    \subsection{Afficher un terrain à partir d'une height map}
    Pour générer le terrain à partir de la height map, il suffit d'appliquer la valeur de la heightmap lors de la génération des vertices :
        \lstinputlisting[language=C++, caption=Selection de la height map, firstline=146, lastline=151, firstnumber=146]{../geometryengine.cpp}
	L'élément important étant la valeur y, caractérisée par la fonction \texttt{qGray()}, qui va récupérer un pixel d'une \texttt{QImage}, dans notre cas qui s'appelle heightMap, et qu'on utilise la fonction \texttt{QImage::pixel(x, y)} pour récupérer la valeur d'un pixel en RGB aux coordonnées fournies.\\
	La fonction \texttt{qGray()} va convertir la valeur en niveaux de gris, et par conséquent donner la hauteur (qui est ici divisée par 50 pour éviter d'avoir des valeurs trop élevées).
	\image{heightmap_terrain.png}{Résultat d'un terrain qui basé sur une heightmap}
	 \image{heightmap.png}{Heightmap appliquée}
	 
	
	\pagebreak
	\section{Question 2}
	\subsection{Modifiez votre fonction d'affichage pour regarder le terrain sous un angle de 45 degrés.}
		Pour regarder le terrain sous un certain angle, nous avons plusieurs possibilités :
		\begin{itemize}
			\item Incliner le terrain
			\item Incliner la caméra
			\item Incliner les deux
		\end{itemize}
		
		Dans notre cas, nous allons uniquement incliner le terrain avec une matrice de rotation, soit un Quaternion :
		\lstinputlisting[language=C++, caption=Rotation du terrain, firstline=231, lastline=232, firstnumber=231]{../mainwidget.cpp}
		\image{avantInclinaison.png}{Avant l'inclinaison}
		\image{apresInclinaison.png}{Après l'inclinaison}
		
	\subsection{Faire tourner le terrain autour de son origine avec une vitesse constante.}
		Lorsqu'on fait glisser le terrain avec la souris, on arrive à faire tourner le terrain, avec une décélération progressive, qui est lié à la friction, pour cela, on constate que tout se passe dans \textit{mainwidget.cpp}, dans les fonctions \texttt{paintGL()} pour appliquer le rendu et \texttt{timerEvent(QTimerEvent*)} qui va appliquer la décélération, et donc une action sur un temps constant.\\
		Cependant, il faut appliquer un mouvement de base, du coup, lors de la mise à jour, on le forcera tout le temps à tourner autour de l'axe Z, avec une vitesse qu'on réglera avec la variable \texttt{angularSpeed}.

	\pagebreak	
	\section{Question 3}
	\subsection{Comment est contrôlée la màj du terrain dans MainWidget ?}
	La mise à jour du terrain s'effectue dans la fonction héritée de QObject :  \textit{timerEvent(QTimerEvent *)}.
	
	\subsection{À quoi sert la classe QTimer ? Comment fonctionne-t-elle ?}
	D'après la documentation du site de Qt\footnote{\url{http://doc.qt.io/qt-5/qtimer.html}} :\\
	"\textit{The QTimer class provides repetitive and single-shot timers.}"\\
	QTimer fournit donc des timers dont on fournit la durée, avec un signal à produire à la fin du timer. On peut donc s'en servir par exemple pour contrôler la fréquence de rafraîchissement de la fenêtre, ou bien scripter une scène (par exemple, on a 30 secondes pour atteindre la sortie sinon la porte sera verrouillée).
	 
	
	\subsection{Modifier le constructeur de la classe MainWidget pour qu'il prenne en paramètre la fréquence de mise à jour.}
	Lors du constructeur, j'ai rajouté un paramètre \texttt{int \_fps} avec une variable en plus pour la stocker, où on assignera le paramètre au timer. Ensuite, il suffit d'appliquer le framerate dans \texttt{MainWidget::initializeGL()} et plus précisément sur \texttt{timer.start()}.
	
	\subsection{Modifier votre programme principal pour afficher votre terrain dans quatre fenêtres différentes.}
	Dans \textit{main.cpp}, il suffit de déclarer 4 MainWidget différents puis de les afficher :
			\lstinputlisting[language=C++, caption=Déclaration et affichage des fenêtres, firstline=70, lastline=78, firstnumber=70]{../main.cpp}
	On constate que les 4 fenêtres n'effectuent pas une rotation en même temps, où à 1 fps elle semble "laguer" et pour 100 et 1000fps elle semble fluide.
	\hfill
	\pagebreak
	Cependant, à partir d'au dessus de 60fps, il semblerait que les vitesses supérieures soient bloqués à un framerate lié à l'écran, cela peut être dû à la synchro verticale que les paramètres de l'ordinateur forcent.
	
	\subsection{Utiliser les flèches pour modifier les vitesses de rotations de votre terrain}
	Dans la fonction \texttt{keyPressEvent(QKeyEvent *e)}, il suffit de rajouter la détection de vitesse (dans mon cas, on utilisera + et - pour augmenter et réduire la vitesse) :
	\lstinputlisting[language=C++, caption=Gestion de la vitesse, firstline=262, lastline=263, firstnumber=262]{../mainwidget.cpp}
	On constate que chaque événement va récupérer la nouvelle information selon la fenêtre sélectionnée.
    
    \pagebreak
    \section{Questions bonus}
    \subsection{Texturer le terrain en utilisant des couleurs}
    Comme pour le précédent
    
    \subsection{Jouer avec la lumière}
\end{document}